
\documentclass[a4paper, 10pt]{article}

\usepackage[english]{babel}
\usepackage[T1]{fontenc}
\usepackage[utf8]{inputenc}
\usepackage{textcomp}

\setlength{\marginparwidth}{2cm}

\usepackage{comment}
\usepackage{todonotes}

\usepackage{amsmath}
\usepackage{amssymb}

\usepackage{xcolor}
\usepackage{graphicx}
\graphicspath{ {./img/} }

\usepackage{hyperref}

\usepackage{enumitem}

\usepackage{listings}
\usepackage{color}

\definecolor{dkgreen}{rgb}{0,0.6,0}
\definecolor{gray}{rgb}{0.5,0.5,0.5}
\definecolor{mauve}{rgb}{0.58,0,0.82}

\lstset{frame=tb,
  language=Python,
  aboveskip=3mm,
  belowskip=3mm,
  showstringspaces=false,
  columns=flexible,
  basicstyle={\small\ttfamily},
  numbers=none,
  numberstyle=\tiny\color{gray},
  keywordstyle=\color{blue},
  commentstyle=\color{dkgreen},
  stringstyle=\color{mauve},
  breaklines=true,
  breakatwhitespace=true,
  tabsize=3
}

\title{Homework Assignment N°2}
\author{BML36\\Thibault Douzon\\Rajavarman Mathivanan}
\date{September 12th, 2018}


\begin{document}
\maketitle

\pagebreak

\tableofcontents
\pagebreak

\section{Exercise 2: Logistic classification \& discrimination}
From now we will use $\sigma(x)=\frac{1}{1+e^{-x}}$
\subsection{Part a}
\begin{itemize}[label=$\square$]
    \item Initialize $w_0$ ?
    \begin{enumerate}
        \item Some fixed $w_0$ like $\begin{bmatrix}0 & 0 & \cdots & 1\end{bmatrix}$
        \item The result of computation around the dataset like the \\mean: $w_0 = \frac{1}{N}\sum_{i=1}^N x_i$, 
        concatenated with a constant.
        \item A random vector
    \end{enumerate}
    Any vector except the null vector and the multiples of $\begin{bmatrix}1 & 0 & \cdots &0\end{bmatrix}$ is suitable to initialize $w_0$, 
    \item How to learn: for batch learning use this equation at each step
    $$
    w_{n+1} = w_n - \eta \nabla E(w_n) = w_n - \eta \sum_{n=1}^{N}\left(y(n)-t_n\right)x_n
    $$
    \item How to stop the iterative process ?
    \begin{enumerate}
    \item Stop when the norm of the difference vector is low: $\Delta_n = \frac{\left\Vert w_{n+1} - w_n\right\Vert}{\left\Vert w_n \right\Vert} < \epsilon$
    \\
    This is a commonly used criterion that stops the process when the steps we take are getting small compared to our current result.
    \item Stop after fixed number of iteration
    \\
    This ensures we won't enter in a infinite non-convergent process. 
    \item Stop when a threshold error is reached: $E(w_n) < \epsilon $
    \\
    This is actually a bad idea because most of the time we can't be certain it is possible to reach such threshold on the error.
    It would result in an infinite process.
    \end{enumerate}
    In a batch version, we can use criteria 1 and 2 together and stop whenever one of 
    the criteria is reached.
    \\
    In a stochastic version, criterium 1 is not applicable because it would stop the learning process whenever a well classified data
    is picked for an iteration.
\end{itemize}

Our algorithm goes as follows:
\begin{enumerate}
    \item Chose $\epsilon$, $N$ and $\eta$ respectivelly for precision, maximum number of iterations and speed convergency.
    \item Set current error $\Delta$ to $+\infty$ and $n$ to $0$
    \item Chose the initial discriminant: $w_{current} = \begin{bmatrix}0 & 0 & \cdots & 1\end{bmatrix}$.
    \item While $\Delta > \epsilon \wedge n < N$ do
    \begin{enumerate}
        \item Compute and store next discriminant $w_{next}$:
$$
w_{next} = w_{current} - \eta \sum_{n=1}^{N}\left(\sigma({w_{current}}^\top x_n)-t_n\right)x_n
$$
        \item Compute and store the new error $\Delta$:
$$
\Delta = \frac{\left\Vert w_{next} - w_{current}\right\Vert}{\left\Vert w_{current} \right\Vert}
$$
        \item Prepare for next iteration: store $w_{next}$ in place of $w_{current}$ and increment $n$
    \end{enumerate}
    \item If $\Delta > \epsilon$, it means we have not converged enough towards the limit. We should consider increasing N OR using another algorithm for convergence (eg. Newton-Raphson)
    \item Result is stored in $w_{current}$, number of steps in $n$. 
\end{enumerate}

\subsection{Part b}
First important thing to notice is that the point $x=\begin{bmatrix}-1 & 1\end{bmatrix}$ is missclassified.
We define $\bar{x} = \begin{bmatrix}1 & -1 & 1\end{bmatrix}$ thus it means that when we compute 
$w^\top \bar{x}$ the sign of the result is incorrect.
\\
In our case, $w^\top \bar{x} = 1$, thus the real class of $x$ is $0$.
\\
The formula to update the weights is the following:
$$
w_{new} = w - \eta \sigma(w^\top \bar{x})\bar{x}
$$
We get the following result:
$$
w_{new} \approx \begin{bmatrix}0.5614 & 2.4386 & 1.5614\end{bmatrix}
$$



\section{Exercise 4: MCQ}
\subsection{First MCQ}
This first question tests if the candidate knows how to compute the new
discriminant from the previous one, using gradient descent.
\\
Q: Which of the following equations can be used to update the discriminant $w$ using
the method of gradient descent ?
\\
\begin{enumerate}
    \item $w_{n+1} - w_{n} =  \eta \nabla E(w_{n})$
    \item $w_{n+1} = w_{n} - \eta \nabla E(w_{n})$
    \item $w_{n} + \eta \nabla E(w_{n-1}) = w_{n-1} $
    \item $w_{n} = \eta w_{n} - \nabla E(w_{n+1})$
\end{enumerate}
\subsection{Second MCQ}
\end{document}

\begin{lstlisting}
    # Some python code
\end{lstlisting}